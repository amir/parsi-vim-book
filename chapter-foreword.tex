\chapter{پیشگفتار}
عالیه! شما واقعا دارید این متن رو می‌خونید. این شمارو در یکی از این ۳ دسته قرار میده:
\begin{enumerate}
\item یک دانشجو/برنامه‌نویس/مدیرسیستم که مجبورش کردند استفاده از Vim رو یاد بگیره،
\item یکی که به صورت کاملا اتفاقی این مستند رو دریافت کرده،‌ و از روی کنجکاوی داره نگاهی بهش میندازه،
\item یکی که واقعا دلش می‌خواد از این به بعد مثل یک مرد از Vim به عنوان ویرایشگر استفاده کنه.
\end{enumerate}


من تقریبا در تمام دورانی که به طور جدی برنامه‌نویسی کردم، Vim رو به عنوان ویراشگر مورد استفاده قرار دادم. در طول این مدت هم کم برخوردهای عجیب و غریب از دوستان و اطرافیان ندیدم، چه اونهایی که اعتقاد داشتند تا Emacs هست، دلیلی برای استفاده-از/وجود Vim نیست و چه اونهایی که اعتقاد داشتند برنامه‌نوشتن با کمتر از Eclipse یا Studio Visual بیشتر به یک شوخی شباهت داره تا واقعیت.

Vim دارای یک خود‌آموز توکار\Footnote{Built-in tutorial}  مناسب کاربران مبتدی و تازه‌کار است، علاوه براین دارای مستندات جامعی می‌باشد که معمولا کاربر را از هر منبع آموزشی دیگری بی‌نیاز می‌کند، اما با این وجود کتابها و نوشته‌های آموزشی بسیاری در مورد Vim وجود دارد، و شما نیز در حال خواندن یکی از آنها می‌باشید! توصیه من این است که از این کتاب به عنوان یک منبع برای یادگیری مفاهیم اولیه Vim استفاده کرده و پس از آن سعی کنید که جواب سوالات خود را در مستندات Vim پیدا کنید.

در این کتاب\footnote{مقاله‌آموزشی، راهنما یا هرچیزه دیگه} من سعی می‌کنم چیزهایی را که هر کاربر Vim برای انجام کارهای روزانه‌اش احتیاج دارد آموزش دهم. فرض بر این است که شما Vim را نصب شده و آماده بر روی سیستم خود دارید\footnote{Vim برروی تمام سیستمهای لینوکس امروزی به صورت پیش فرض وجود دارد، و فرض بر این است که شما از لینوکس استفاده می کنید، هرچند که نصب آن بر روی ویندوز به سادگی امکان پذیر است.}. \\

این کتاب با استفاده از \mbox{\lr{\TeX\\}} و \mbox{\lr{\XeTeX\\}} نوشته شده است، توزیع تکلایو ۲۰۰۸ \footnote{قابل دریافت از: http://tug.org/texlive/}تمامی نیازهای نوشتن این کتاب (از جمله بسته \mbox{\lr{\XePersian\\}} که برای پارسی نویسی استفاده شده\footnote{با تشکر ویژه از گروه فارسی-لاتک،‌ آقایان: مصطفی واحدی، وفا خلیقی و مهدی امیدعلی}) را برآورده کرده.\\
\newpage
{\sols اگر پس از مطالعه این کتاب آن‌را سودمند یافته و احساس کردید که چیز جدیدی یاد گرفته‌اید، لطفا فراموش نکنید که هستند کودکان بی‌سرپرستی که تشنه یادگیری هستند و چه‌بسا در صورت فراهم آمدن امکانات افرادی به مراتب تاثیرگذارتر از ما در آینده باشند، اما از بد روزگار و به دلیل شرایط بد مالی امکان ادامه‌تحصیل را ندارند. پس کمک به کودکان بی‌سرپرست را فراموش نکنید.}
